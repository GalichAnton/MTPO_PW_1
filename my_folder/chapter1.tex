\chapter{Проектирование приложения} \label{ch:design}

Данная глава посвящена проектированию приложения для топологической сортировки графов.

\section{Функциональные требования и интерфейс} \label{sec:usecase}

Диаграмма вариантов использования (Use-Case) отображает функциональные требования к системе с точки зрения пользователя.
На \firef{fig:usecase} представлена диаграмма вариантов использования приложения топологической сортировки.


\begin{figure}[htbp]
    \centering
    \begin{tikzpicture}[
        box/.style={draw, rectangle, minimum width=11cm, minimum height=9cm, thick},
        header/.style={fill=gray!20, font=\bfseries, minimum width=11cm, minimum height=0.8cm},
        actor/.style={draw, rectangle, minimum width=2cm, minimum height=1.2cm, font=\small, align=center},
        action/.style={font=\small}
    ]

        \node[box] (system) at (0,0) {};

        \node[header, anchor=south] at (system.north) {Система топологической сортировки};

        \node[actor] (user) at (-7, 0) {Пользо-\\ватель};

        \node[action, anchor=west] (uc1) at (-4.5, 3.5) {Показать справку};
        \node[action, anchor=west] (uc2) at (-4.5, 2.5) {Ввести граф вручную};
        \node[action, anchor=west] (uc3) at (-4.5, 1.5) {Загрузить граф из JSON-файла};
        \node[action, anchor=west] (uc4) at (-4.5, 0.5) {Выбрать алгоритм (DFS / Kahn)};
        \node[action, anchor=west] (uc5) at (-4.5, -0.5) {Выполнить сортировку};
        \node[action, anchor=west] (uc6) at (-4.5, -1.5) {Сравнить алгоритмы};
        \node[action, anchor=west] (uc7) at (-4.5, -2.5) {Сохранить результат в файл};
        \node[action, anchor=west] (uc8) at (-4.5, -3.5) {Показать текущий граф};

        \draw[->, thick] (user.east) -- (uc1.west);
        \draw[->, thick] (user.east) -- (uc2.west);
        \draw[->, thick] (user.east) -- (uc3.west);
        \draw[->, thick] (user.east) -- (uc4.west);
        \draw[->, thick] (user.east) -- (uc5.west);
        \draw[->, thick] (user.east) -- (uc6.west);
        \draw[->, thick] (user.east) -- (uc7.west);
        \draw[->, thick] (user.east) -- (uc8.west);

    \end{tikzpicture}
    \caption{Диаграмма вариантов использования приложения}
    \label{fig:usecase}
\end{figure}

Система поддерживает восемь основных вариантов использования:
\begin{itemize}
    \item \textbf{Показать справку} --- просмотр справочной информации о программе и алгоритмах;
    \item \textbf{Ввести граф вручную} --- ручной ввод графа с валидацией входных данных;
    \item \textbf{Загрузить граф из JSON-файла} --- загрузка графа из внешнего файла с парсингом;
    \item \textbf{Выбрать алгоритм (DFS / Kahn)} --- выбор алгоритма топологической сортировки;
    \item \textbf{Выполнить сортировку} --- запуск топологической сортировки с проверкой на циклы;
    \item \textbf{Сравнить алгоритмы} --- сравнительный анализ работы обоих алгоритмов;
    \item \textbf{Сохранить результат в файл} --- сохранение результатов в JSON-формате;
    \item \textbf{Показать текущий граф} --- вывод структуры загруженного графа.
\end{itemize}

\textbf{Меню приложения} \label{sec:menu}

Интерфейс приложения реализован в виде консольного меню.
На \firef{fig:menu} представлен внешний вид главного меню программы.

\begin{figure}[htbp]
    \centering
    \begin{tcolorbox}[
        colback=black!5,
        colframe=black!75,
        width=0.85\textwidth,
        arc=0mm,
        boxrule=1pt,
        fontupper=\ttfamily\small
    ]
        \begin{center}
            \textbf{ТОПОЛОГИЧЕСКАЯ СОРТИРОВКА ГРАФА v1.0}
        \end{center}
        \tcblower
        \begin{tabbing}
            \hspace{1em}\=\kill
            \>1. Справка о программе\\
            \>2. Ввести граф вручную\\
            \>3. Загрузить граф из JSON-файла\\
            \>4. Показать текущий граф\\
            \>5. Выполнить топологическую сортировку (DFS)\\
            \>6. Выполнить топологическую сортировку (Кан)\\
            \>7. Сравнить результаты обоих алгоритмов\\
            \>8. Сохранить результат в JSON-файл\\
            \>0. Выход
        \end{tabbing}
    \end{tcolorbox}
    \caption{Главное меню приложения}
    \label{fig:menu}
\end{figure}

Меню предоставляет пользователю полный набор функций для работы с графами: от ввода данных до сохранения результатов.
Навигация осуществляется путём ввода номера соответствующего пункта меню.

\textbf{Справочная информация} \label{sec:help}

Справочная система приложения содержит теоретические сведения об алгоритмах и формате данных.
Содержимое справки представлено на \firef{fig:help}.



\begin{figure}[htbp]
    \centering
    \begin{tcolorbox}[
        reset,
        colback=blue!5,
        colframe=blue!50!black,
        width=0.9\textwidth,
        arc=0mm,
        boxrule=1pt,
        title={\centering\textbf{СПРАВКА}},
        fontupper=\small,
        breakable=false
    ]
        \textbf{ОПИСАНИЕ:}
        \begin{itemize}[nosep, leftmargin=1.5em]
            \item Программа выполняет топологическую сортировку
            \item ориентированного ациклического графа (DAG)
            \item Результат — линейный порядок вершин, где для
            \item каждого ребра u → v вершина u идёт перед v
        \end{itemize}

        \vspace{0.5em}
        \textbf{АЛГОРИТМЫ:}
        \begin{itemize}[nosep, leftmargin=1.5em]
            \item DFS — на основе поиска в глубину
            \item Kahn — на основе очереди с входящими степенями
        \end{itemize}

        \vspace{0.5em}
        \textbf{ФОРМАТ JSON-ФАЙЛА:}
        \begin{verbatim}
{
    "vertices": ["A", "B", "C"],
    "edges": [
        {"from": "A", "to": "B"},
        {"from": "B", "to": "C"}
    ]
}
        \end{verbatim}

        \vspace{0.5em}
        \textbf{РЕКОМЕНДАЦИИ:}
        \begin{itemize}[nosep, leftmargin=1.5em]
            \item Убедитесь, что граф не содержит циклов
            \item Все вершины в рёбрах должны существовать
            \item Для больших графов используйте загрузку из файла
            \item Используйте сравнение алгоритмов для анализа
        \end{itemize}
    \end{tcolorbox}
    \caption{Справочная информация приложения}
    \label{fig:help}
\end{figure}
\FloatBarrier



\section{Структура проекта} \label{sec:structure}

Проект организован согласно стандартной структуре Maven-проекта.

Проект разделён на следующие пакеты:
\begin{itemize}
    \item \texttt{model} --- классы предметной области (граф, результаты);
    \item \texttt{algorithm} --- реализации алгоритмов топологической сортировки;
    \item \texttt{io} --- ввод-вывод данных (JSON, консоль);
    \item \texttt{util} --- валидация данных;
\end{itemize}

Тестовый код организован зеркально основному коду, что обеспечивает удобство навигации и поддержки.

\section{Выводы} \label{sec:design-conclusion}

В данной главе выполнено проектирование приложения для топологической сортировки графов. Разработана диаграмма вариантов использования, определяющая восемь основных сценариев взаимодействия пользователя с системой. Спроектирован консольный интерфейс с интуитивно понятным меню. Определена модульная структура проекта, обеспечивающая разделение ответственности между компонентами и удобство тестирования.
\clearpage