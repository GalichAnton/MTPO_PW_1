\chapter{Проектирование приложения} \label{ch:design}

Данная глава посвящена проектированию приложения для топологической сортировки графов. В параграфе \ref{sec:usecase} представлена диаграмма вариантов использования системы. Параграф \ref{sec:menu} описывает интерфейс меню приложения. В параграфе \ref{sec:help} приведена справочная информация о программе. Параграф \ref{sec:structure} раскрывает структуру проекта.

\section{Use-Case диаграмма} \label{sec:usecase}

Диаграмма вариантов использования (Use-Case) отображает функциональные требования к системе с точки зрения пользователя.
На \firef{fig:usecase} представлена диаграмма вариантов использования приложения топологической сортировки.


\begin{figure}[htbp]
    \centering
    \begin{tikzpicture}[
        actor/.style={circle, draw, minimum size=1cm, font=\small},
        usecase/.style={ellipse, draw, minimum width=3.2cm, minimum height=0.9cm, align=center, font=\footnotesize},
        include/.style={ellipse, draw, dashed, minimum width=2.5cm, minimum height=0.7cm, align=center, font=\scriptsize},
        arrow/.style={->, >=stealth},
        inclarrow/.style={->, >=stealth, dashed}
    ]

% Актор
        \node[actor] (user) at (0,-6.5) {User};

% Use Cases - увеличены вертикальные интервалы
        \node[usecase] (uc1) at (5.5,-1) {UC1: Показать справку};
        \node[usecase] (uc2) at (5.5,-2.8) {UC2: Ввести граф вручную};
        \node[usecase] (uc3) at (5.5,-4.6) {UC3: Загрузить граф из JSON};
        \node[usecase] (uc4) at (5.5,-6.4) {UC4: Выбрать алгоритм};
        \node[usecase] (uc5) at (5.5,-9) {UC5: Выполнить сортировку};
        \node[usecase] (uc6) at (5.5,-11) {UC6: Вывод в консоль};
        \node[usecase] (uc7) at (5.5,-12.5) {UC7: Сохранить в JSON};
        \node[usecase] (uc8) at (5.5,-13.5) {UC8: Сравнить алгоритмы};

% Include relationships - сдвинуты вправо
        \node[include] (val) at (11,-2.8) {Валидация ввода};
        \node[include] (parse) at (11,-4.6) {Парсинг файла};
        \node[include] (cycle) at (11,-9) {Проверка на циклы};

% Алгоритмы - размещены ниже UC4 с отступом
        \node[include] (dfs) at (4,-7.7) {DFS-based};
        \node[include] (kahn) at (7.5,-7.7) {Алгоритм Кана};

% Связи актора с use case
        \draw[arrow] (user) -- (uc1);
        \draw[arrow] (user) -- (uc2);
        \draw[arrow] (user) -- (uc3);
        \draw[arrow] (user) -- (uc4);
        \draw[arrow] (user) -- (uc5);
        \draw[arrow] (user) -- (uc6);
        \draw[arrow] (user) -- (uc7);
        \draw[arrow] (user) -- (uc8);

% Include связи
        \draw[inclarrow] (uc2) -- (val) node[midway, above, font=\tiny] {<<include>>};
        \draw[inclarrow] (uc3) -- (parse) node[midway, above, font=\tiny] {<<include>>};
        \draw[inclarrow] (uc5) -- (cycle) node[midway, above, font=\tiny] {<<include>>};

% Связи выбора алгоритма
        \draw[inclarrow] (uc4) -- (dfs);
        \draw[inclarrow] (uc4) -- (kahn);

    \end{tikzpicture}
    \caption{Диаграмма вариантов использования приложения}
    \label{fig:usecase}
\end{figure}


Система поддерживает восемь основных вариантов использования:
\begin{itemize}
    \item \textbf{UC1} --- просмотр справочной информации о программе и алгоритмах;
    \item \textbf{UC2} --- ручной ввод графа с валидацией входных данных;
    \item \textbf{UC3} --- загрузка графа из JSON-файла с парсингом;
    \item \textbf{UC4} --- выбор алгоритма сортировки (DFS или Кана);
    \item \textbf{UC5} --- выполнение топологической сортировки с проверкой на циклы;
    \item \textbf{UC6} --- вывод результатов в консоль;
    \item \textbf{UC7} --- сохранение результатов в JSON-файл;
\end{itemize}

\section{Меню приложения} \label{sec:menu}

Интерфейс приложения реализован в виде консольного меню. На \firef{fig:menu} представлен внешний вид главного меню программы.

\begin{figure}[htbp]
    \centering
    \begin{tcolorbox}[
        colback=black!5,
        colframe=black!75,
        width=0.85\textwidth,
        arc=0mm,
        boxrule=1pt,
        fontupper=\ttfamily\small
    ]
        \begin{center}
            \textbf{ТОПОЛОГИЧЕСКАЯ СОРТИРОВКА ГРАФА v1.0}
        \end{center}
        \tcblower
        \begin{tabbing}
            \hspace{1em}\=\kill
            \>1. Справка о программе\\
            \>2. Ввести граф вручную\\
            \>3. Загрузить граф из JSON-файла\\
            \>4. Показать текущий граф\\
            \>5. Выполнить топологическую сортировку (DFS)\\
            \>6. Выполнить топологическую сортировку (Кан)\\
            \>7. Сравнить результаты обоих алгоритмов\\
            \>8. Сохранить результат в JSON-файл\\
            \>0. Выход
        \end{tabbing}
    \end{tcolorbox}
    \caption{Главное меню приложения}
    \label{fig:menu}
\end{figure}

Меню предоставляет пользователю полный набор функций для работы с графами: от ввода данных до сохранения результатов. Навигация осуществляется путём ввода номера соответствующего пункта меню.

\section{Справочная информация} \label{sec:help}

Справочная система приложения содержит теоретические сведения об алгоритмах и формате данных.
Содержимое справки представлено на \firef{fig:help}.


\begin{figure}[htbp]
    \centering
    \begin{tcolorbox}[
        colback=blue!5,
        colframe=blue!50!black,
        width=0.9\textwidth,
        arc=0mm,
        boxrule=1pt,
        title={\centering\textbf{СПРАВКА}},
        fontupper=\small
    ]
        \textbf{ОПИСАНИЕ:}
        \begin{itemize}[nosep, leftmargin=1.5em]
            \item Программа выполняет топологическую сортировку
            \item ориентированного ациклического графа (DAG)
            \item Результат — линейный порядок вершин, где для
            \item каждого ребра u → v вершина u идёт перед v
        \end{itemize}

        \vspace{0.5em}
        \textbf{АЛГОРИТМЫ:}
        \begin{itemize}[nosep, leftmargin=1.5em]
            \item DFS — на основе поиска в глубину
            \item Kahn — на основе очереди с входящими степенями
        \end{itemize}

        \vspace{0.5em}
        \textbf{ФОРМАТ JSON-ФАЙЛА:}
        \begin{verbatim}
            {
                "vertices": ["A", "B", "C"],
                "edges": [
                    {"from": "A", "to": "B"},
                    {"from": "B", "to": "C"}
                ]
            }
        \end{verbatim}

        \vspace{0.5em}
        \textbf{РЕКОМЕНДАЦИИ:}
        \begin{itemize}[nosep, leftmargin=1.5em]
            \item Убедитесь, что граф не содержит циклов
            \item Все вершины в рёбрах должны существовать
            \item Для больших графов используйте загрузку из файла
            \item Используйте сравнение алгоритмов для анализа
        \end{itemize}
    \end{tcolorbox}
    \caption{Справочная информация приложения}
    \label{fig:help}
\end{figure}


\section{Структура проекта} \label{sec:structure}

Проект организован согласно стандартной структуре Maven-проекта с разделением на слои. Структура каталогов представлена на \firef{fig:structure}.

Проект разделён на следующие пакеты:
\begin{itemize}
    \item \texttt{model} --- классы предметной области (граф, результаты);
    \item \texttt{algorithm} --- реализации алгоритмов топологической сортировки;
    \item \texttt{io} --- ввод-вывод данных (JSON, консоль);
    \item \texttt{util} --- валидация данных;
\end{itemize}

Тестовый код организован зеркально основному коду, что обеспечивает удобство навигации и поддержки.

\section{Выводы} \label{sec:design-conclusion}

В данной главе выполнено проектирование приложения для топологической сортировки графов. Разработана диаграмма вариантов использования, определяющая восемь основных сценариев взаимодействия пользователя с системой. Спроектирован консольный интерфейс с интуитивно понятным меню. Определена модульная структура проекта, обеспечивающая разделение ответственности между компонентами и удобство тестирования.