\chapter*{Заключение}
\addcontentsline{toc}{chapter}{Заключение}

В ходе практической работы по теме <<Разработка кода через модульное тестирование>> достигнута поставленная цель --- изучены и практически применены принципы модульного тестирования и разработки через тестирование (TDD). Решены все задачи: изучена теория модульного тестирования, проанализированы инструменты, спроектирована архитектура приложения, разработана тестовая документация и реализован полный набор тестов.

\textbf{Результаты проектирования приложения.} Разработано консольное приложение для топологической сортировки ориентированных ациклических графов (DAG) с реализацией двух алгоритмов: DFS (поиск в глубину) и алгоритма Кана (на основе входящих степеней). Архитектура приложения включает модель данных (классы Graph, SortResult, ComparisonResult), интерфейс TopologicalSortAlgorithm с двумя реализациями, компоненты ввода-вывода (JsonGraphHandler) и валидации (SortValidator). Приложение поддерживает загрузку графов из JSON-файлов, ручной ввод, сравнение алгоритмов и сохранение результатов.

\textbf{Результаты проектирования тестов.} Разработана тестовая документация с использованием четырёх методов: Equivalence Partitioning (4 теста), Boundary Value Analysis (6 тестов), Decision Table Testing (6 тестов) и Branch Testing (5 тестов). Определено 16 проверяемых условий, распределённых по категориям: входные данные, выходные данные и внутренняя обработка. Общие трудозатраты на разработку тестовой документации составили 17 человеко-часов.

\textbf{Преимущества выбранной библиотеки JUnit~5:}
\begin{itemize}
    \item современная модульная архитектура (JUnit Platform, Jupiter, Vintage);
    \item богатый набор аннотаций для управления жизненным циклом тестов;
    \item встроенная поддержка параметризованных тестов (@ValueSource, @CsvSource, @MethodSource);
    \item вложенные тестовые классы (@Nested) для логической группировки;
    \item улучшенный API assertions с поддержкой лямбда-выражений;
    \item отличная интеграция с IDE и инструментами сборки (Maven, Gradle);
    \item активное развитие и поддержка сообществом.
\end{itemize}

\textbf{Недостатки JUnit~5:}
\begin{itemize}
    \item более высокий порог входа по сравнению с JUnit~4 из-за модульной архитектуры;
    \item необходимость дополнительных библиотек (Hamcrest, Mockito) для полноценного тестирования;
    \item отсутствие встроенной поддержки мутационного тестирования.
\end{itemize}

\textbf{Количество и качество разработанных тестов.} Всего реализовано более 125 тестовых методов в 5 тестовых классах. Объём тестового кода составил около 3500 строк. Метрики покрытия: 94\% по строкам (Line Coverage), 90\% по ветвям (Branch Coverage), 99\% по методам (Method Coverage). Мутационное тестирование с использованием Pitest показало Mutation Score 82\%, что превышает пороговое значение 80\%. Использованы различные техники: 6+ методов assertions, 2+ метода assumptions, 3+ вида мокирования, 4+ категории Hamcrest-матчеров.

\textbf{Преимущества использования TDD в контексте работы:}
\begin{itemize}
    \item высокое качество кода --- тесты выявляли ошибки на ранних этапах разработки;
    \item улучшенная архитектура --- необходимость тестируемости привела к слабой связанности компонентов;
    \item тесты как документация --- тестовые случаи служат спецификацией поведения системы;
    \item уверенность при рефакторинге --- изменения кода подтверждаются прохождением тестов;
    \item раннее обнаружение циклов --- тесты сразу проверяли корректность обнаружения циклов в графах;
    \item сравнение алгоритмов --- тесты обеспечили верификацию эквивалентности результатов DFS и Кана.
\end{itemize}

\textbf{Недостатки использования TDD:}
\begin{itemize}
    \item увеличение времени разработки на начальном этапе;
    \item необходимость переработки тестов при изменении требований;
    \item избыточность тестов для простых методов (геттеры, сеттеры);
    \item сложность тестирования граничных случаев алгоритмов на графах;
    \item дополнительные усилия на поддержание актуальности тестового набора.
\end{itemize}
