\chapter*{Заключение}
\addcontentsline{toc}{chapter}{Заключение}

В ходе практической работы по теме <<Разработка кода через модульное тестирование>> достигнута поставленная цель --- изучены и практически применены принципы модульного тестирования. Решены все задачи: изучена теория, проанализированы инструменты, разработаны тестовые сценарии, проанализировано влияние тестирования на качество кода, сформулированы рекомендации.

Основные выводы:
\begin{itemize}
    \item Модульное тестирование эффективно повышает качество и надежность кода
    \item Тесты служат документацией и упрощают рефакторинг
    \item Внедрение требует изменения подходов, но окупается сокращением времени на отладку
    \item Важна не только техническая реализация, но и культура тестирования
\end{itemize}

Практическая значимость: разработанные материалы могут быть использованы для обучения и внедрения тестирования в реальных проектах. Дальнейшая работа может быть направлена на изучение интеграционного тестирования и CI/CD автоматизации.