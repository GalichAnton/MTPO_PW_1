\chapter*{Введение}
\addcontentsline{toc}{chapter}{Введение}

Современная разработка программного обеспечения предъявляет высокие требования к качеству, надежности и сопровождаемости кода.
В условиях быстрой смены требований и необходимости частых обновлений программных продуктов особую актуальность приобретают методы и практики, обеспечивающие стабильность работы системы при внесении изменений.

\textbf{Актуальность исследования} обусловлена тем, что модульное тестирование (unit testing) является фундаментальной практикой в современных процессах разработки программного обеспечения.

\textbf{Объект исследования} --- процесс разработки программного обеспечения с использованием методологий тестирования.

\textbf{Предмет исследования} --- методы и практики модульного тестирования в контексте разработки и сопровождения программного кода.

\textbf{Цель практической работы} --- изучить и применить на практике принципы модульного тестирования для разработки качественного программного кода.

\textbf{Практическая часть работы} будет посвящена тестированию приложения, реализующего алгоритмы топологической сортировки ориентированного ациклического графа: на основе поиска в глубину (DFS) и алгоритма Кана.


%% Вспомогательные команды - Additional commands
%\newpage % принудительное начало с новой страницы, использовать только в конце раздела
%\clearpage % осуществляется пакетом <<placeins>> в пределах секций
%\newpage\leavevmode\thispagestyle{empty}\newpage % 100 % начало новой строки