\chapter*{Введение}
\addcontentsline{toc}{chapter}{Введение}

Современная разработка программного обеспечения предъявляет высокие требования к качеству, надежности и сопровождаемости кода. В условиях быстрой смены требований и необходимости частых обновлений программных продуктов особую актуальность приобретают методы и практики, обеспечивающие стабильность работы системы при внесении изменений.

\textbf{Актуальность исследования} обусловлена тем, что модульное тестирование (unit testing) является фундаментальной практикой в современных процессах разработки программного обеспечения, однако его эффективное внедрение и использование остается проблемой для многих команд разработчиков. Недостаточное понимание принципов модульного тестирования, неправильная организация тестов и отсутствие культуры тестирования приводят к снижению качества кода, увеличению количества ошибок и росту затрат на поддержку программных продуктов.

\textbf{Объект исследования} --- процесс разработки программного обеспечения с использованием методологий тестирования.

\textbf{Предмет исследования} --- методы и практики модульного тестирования в контексте разработки и сопровождения программного кода.

\textbf{Цель практической работы} --- изучить и применить на практике принципы модульного тестирования для разработки качественного программного кода.

\textbf{Практическая часть работы} будет посвящена тестированию приложения, реализующего алгоритмы топологической сортировки ориентированного ациклического графа: на основе поиска в глубину (DFS) и алгоритма Кана.

Для достижения поставленной цели необходимо решить следующие \textbf{задачи}:
\begin{enumerate}
    \item Изучить теоретические основы модульного тестирования и его роль в процессе разработки ПО.
    \item Рассмотреть современные фреймворки и инструменты для модульного тестирования.
    \item Разработать тестовые сценарии для существующего кода с использованием выбранного фреймворка.
    \item Проанализировать влияние модульного тестирования на качество и сопровождаемость кода.
\end{enumerate}

%% Вспомогательные команды - Additional commands
%\newpage % принудительное начало с новой страницы, использовать только в конце раздела
%\clearpage % осуществляется пакетом <<placeins>> в пределах секций
%\newpage\leavevmode\thispagestyle{empty}\newpage % 100 % начало новой строки