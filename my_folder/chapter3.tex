\chapter{Разработка тестовой документации} \label{ch3}

В данной главе представлена разработка тестовой документации для приложения топологической сортировки, реализующего алгоритмы DFS и Кана.
Сначала определены проверяемые условия (test cover items), затем для каждого метода тестирования разработаны контрольные примеры (test cases).
В завершение главы приведён сравнительный анализ использованных методов с оценкой трудозатрат и рекомендациями по оптимизации тестового набора~\cite{iso29119-1, iso29119-4}.

\section{Проверяемые условия (Test Cover Items)} \label{ch3:items}

Проверяемые условия определяют, какие аспекты системы подлежат тестированию.
Они разделены на три категории: входные данные, выходные данные и внутренняя обработка.

\textbf{Входные данные}

В таблице~\ref{tab:input_items} представлены проверяемые условия для входных данных приложения.

\begin{table}[htbp]
    \centering
    \caption{Проверяемые условия для входных данных}
    \label{tab:input_items}
    \begin{tabular}{|c|l|p{7cm}|}
        \hline
        \textbf{ID} & \textbf{Условие} & \textbf{Описание} \\
        \hline
        IN-01 & Пустой граф & vertices = [], edges = [] \\
        \hline
        IN-02 & Одна вершина без рёбер & vertices = [1], edges = [] \\
        \hline
        IN-03 & Линейный граф & A $\rightarrow$ B $\rightarrow$ C $\rightarrow$ D \\
        \hline
        IN-04 & Граф с несколькими корнями & Несколько вершин без входящих рёбер \\
        \hline
        IN-05 & Граф с циклом & Недопустимый ввод --- должна быть ошибка \\
        \hline
        IN-06 & Некорректный JSON & Невалидный формат файла \\
        \hline
        IN-07 & Несуществующий файл & Путь к файлу, которого нет \\
        \hline
        IN-08 & Большой граф & 1000+ вершин \\
        \hline
    \end{tabular}
\end{table}

\subsection*{Выходные данные}

В таблице~\ref{tab:output_items} представлены проверяемые условия для выходных данных.

\begin{table}[htbp]
    \centering
    \caption{Проверяемые условия для выходных данных}
    \label{tab:output_items}
    \begin{tabular}{|c|l|p{7cm}|}
        \hline
        \textbf{ID} & \textbf{Условие} & \textbf{Описание} \\
        \hline
        OUT-01 & Корректный порядок & Каждое ребро u$\rightarrow$v: u предшествует v в результате \\
        \hline
        OUT-02 & Все вершины присутствуют & Результат содержит все вершины графа \\
        \hline
        OUT-03 & Сообщение об ошибке при цикле & Информативное сообщение \\
        \hline
        OUT-04 & JSON-формат вывода & Валидный JSON в файле \\
        \hline
    \end{tabular}
\end{table}
\FloatBarrier

\subsection*{Внутренняя обработка}

В таблице~\ref{tab:proc_items} представлены проверяемые условия для внутренней обработки.

\begin{table}[htbp]
    \centering
    \caption{Проверяемые условия для внутренней обработки}
    \label{tab:proc_items}
    \begin{tabular}{|c|l|p{7cm}|}
        \hline
        \textbf{ID} & \textbf{Условие} & \textbf{Описание} \\
        \hline
        PROC-01 & DFS посещает все вершины & Все вершины обработаны \\
        \hline
        PROC-02 & Kahn корректно считает in-degree & Счётчик входящих рёбер \\
        \hline
        PROC-03 & Детектирование цикла в DFS & Состояния: WHITE/GRAY/BLACK \\
        \hline
        PROC-04 & Детектирование цикла в Kahn & Не все вершины в результате \\
        \hline
    \end{tabular}
\end{table}
\FloatBarrier

\section{Equivalence Partitioning (Разбиение на классы эквивалентности)} \label{ch3:ep}

В таблице~\ref{tab:ep_classes} представлены выделенные классы эквивалентности для входных данных приложения.

\begin{table}[htbp]
    \centering
    \caption{Классы эквивалентности входных данных}
    \label{tab:ep_classes}
    \begin{tabular}{|c|p{8cm}|c|}
        \hline
        \textbf{Класс} & \textbf{Описание} & \textbf{Валидный?} \\
        \hline
        EP1 & Пустой граф & Да \\
        \hline
        EP2 & DAG (ациклический) & Да \\
        \hline
        EP3 & Граф с циклом & Нет \\
        \hline
        EP4 & Некорректный формат & Нет \\
        \hline
    \end{tabular}
\end{table}
\FloatBarrier

На основе выделенных классов эквивалентности разработаны контрольные примеры, представленные в таблице~\ref{tab:ep_cases}.

\begin{table}[htbp]
    \centering
    \caption{Контрольные примеры для Equivalence Partitioning}
    \label{tab:ep_cases}
    \begin{tabular}{|c|c|p{5.5cm}|p{3.5cm}|}
        \hline
        \textbf{TC ID} & \textbf{Класс} & \textbf{Входные данные} & \textbf{Ожидаемый результат} \\
        \hline
        TC-EP-01 & EP1 & vertices: пусто; edges: пусто & sorted: пусто \\
        \hline
        TC-EP-02 & EP2 & vertices: [A, B, C]; edges: [A$\to$B, B$\to$C] & sorted: [A, B, C] \\
        \hline
        TC-EP-03 & EP3 & vertices: [A, B]; edges: [A$\to$B, B$\to$A] & Ошибка: цикл \\
        \hline
        TC-EP-04 & EP4 & Невалидный JSON & Ошибка: некорректный формат \\
        \hline
    \end{tabular}
\end{table}
\FloatBarrier

\section{Boundary Value Analysis (Анализ граничных значений)} \label{ch3:bv}

Граничные значения для параметров приложения представлены в таблице~\ref{tab:bva_boundaries}.

\begin{table}[htbp]
    \centering
    \caption{Граничные значения параметров}
    \label{tab:bva_boundaries}
    \begin{tabular}{|l|l|}
        \hline
        \textbf{Граница} & \textbf{Значения для теста} \\
        \hline
        Количество вершин & 0, 1, 2, MAX \\
        \hline
        Количество рёбер & 0, 1, n-1, n(n-1)/2 \\
        \hline
    \end{tabular}
\end{table}
\FloatBarrier

Контрольные примеры для граничных значений представлены в таблице~\ref{tab:bva_cases}.

\begin{table}[htbp]
    \centering
    \caption{Контрольные примеры для Boundary Value Analysis}
    \label{tab:bva_cases}
    \begin{tabular}{|c|p{2.5cm}|p{4.5cm}|p{3cm}|}
        \hline
        \textbf{TC ID} & \textbf{Граница} & \textbf{Входные данные} & \textbf{Ожидаемый результат} \\
        \hline
        TC-BVA-01 & 0 вершин & vertices: пусто; edges: пусто & [] \\
        \hline
        TC-BVA-02 & 1 вершина & vertices: [A]; edges: пусто & [A] \\
        \hline
        TC-BVA-03 & 2 вершины, 0 рёбер & vertices: [A, B]; edges: пусто & [A, B] или [B, A] \\
        \hline
        TC-BVA-04 & 2 вершины, 1 ребро & vertices: [A, B]; edges: [A$\to$B] & [A, B] \\
        \hline
        TC-BVA-05 & Полный DAG (3 верш.) & 3 вершины, 3 ребра & Валидный порядок \\
        \hline
        TC-BVA-06 & Большой граф & 1000 вершин линейно & Корректный порядок \\
        \hline
    \end{tabular}
\end{table}
\FloatBarrier

\section{Decision Table Testing (Таблица решений). Условия и действия} \label{ch3:dt}

Для построения таблицы решений выделены следующие условия:
\begin{itemize}
    \item C1: Граф пустой?
    \item C2: Есть цикл?
    \item C3: Файл существует?
    \item C4: JSON валидный?
    \item C5: Алгоритм выбран?
\end{itemize}
\FloatBarrier
Таблица решений представлена в таблице~\ref{tab:dt_rules}.

\begin{table}[htbp]
    \centering
    \caption{Таблица решений}
    \label{tab:dt_rules}
    \begin{tabular}{|l|c|c|c|c|c|c|}
        \hline
        & \textbf{R1} & \textbf{R2} & \textbf{R3} & \textbf{R4} & \textbf{R5} & \textbf{R6} \\
        \hline
        \multicolumn{7}{|l|}{\textbf{Условия}} \\
        \hline
        Граф пустой? & Y & N & N & N & N & N \\
        \hline
        Есть цикл? & -- & N & Y & N & N & N \\
        \hline
        Файл существует? & -- & Y & Y & N & Y & Y \\
        \hline
        JSON валидный? & -- & Y & Y & -- & N & Y \\
        \hline
        Алгоритм выбран? & -- & Y & Y & -- & -- & N \\
        \hline
        \multicolumn{7}{|l|}{\textbf{Действия}} \\
        \hline
        Вернуть пустой результат & X & & & & & \\
        \hline
        Выполнить сортировку & & X & & & & \\
        \hline
        Ошибка: цикл & & & X & & & \\
        \hline
        Ошибка: файл не найден & & & & X & & \\
        \hline
        Ошибка: невалидный JSON & & & & & X & \\
        \hline
        Запросить выбор алгоритма & & & & & & X \\
        \hline
    \end{tabular}
\end{table}
\FloatBarrier

Контрольные примеры на основе таблицы решений представлены в таблице~\ref{tab:dt_cases}.

\begin{table}[htbp]
    \centering
    \caption{Контрольные примеры для Decision Table Testing}
    \small
    \label{tab:dt_cases}
    \begin{tabular}{|c|c|p{7cm}|}
        \hline
        \textbf{TC ID} & \textbf{Правило} & \textbf{Описание} \\
        \hline
        TC-DT-01 & R1 & Пустой граф $\rightarrow$ пустой результат \\
        \hline
        TC-DT-02 & R2 & Валидный DAG $\rightarrow$ сортировка \\
        \hline
        TC-DT-03 & R3 & Граф с циклом $\rightarrow$ ошибка \\
        \hline
        TC-DT-04 & R4 & Несуществующий файл $\rightarrow$ ошибка \\
        \hline
        TC-DT-05 & R5 & Невалидный JSON $\rightarrow$ ошибка \\
        \hline
        TC-DT-06 & R6 & Не выбран алгоритм $\rightarrow$ запрос \\
        \hline
    \end{tabular}
\end{table}
\FloatBarrier

\section{Branch Testing (Тестирование ветвей)} \label{ch3:bt}

Тестирование ветвей относится к методам структурного (white-box) тестирования и требует прохождения каждой ветви условных операторов в коде хотя бы один раз.

Структура ветвей алгоритма DFS представлена следующим образом:
\begin{enumerate}
    \item \texttt{if (graph.isEmpty())} $\rightarrow$ return empty
    \item \texttt{for each vertex}:
    \begin{enumerate}
        \item \texttt{if (not visited)} $\rightarrow$ start DFS
        \begin{enumerate}
            \item \texttt{if (visiting - GRAY)} $\rightarrow$ cycle detected
            \item \texttt{if (visited - BLACK)} $\rightarrow$ skip
            \item \texttt{else} $\rightarrow$ process neighbors
        \end{enumerate}
    \end{enumerate}
\end{enumerate}

Контрольные примеры для покрытия ветвей представлены в таблице~\ref{tab:br_cases}.

\begin{table}[htbp]
    \centering
    \caption{Контрольные примеры для Branch Testing}
    \label{tab:br_cases}
    \begin{tabular}{|c|p{4cm}|p{4cm}|p{3.5cm}|}
        \hline
        \textbf{TC ID} & \textbf{Покрываемая ветвь} & \textbf{Входные данные} & \textbf{Путь} \\
        \hline
        TC-BR-01 & Ветвь 1 (true) & Пустой граф & isEmpty $\rightarrow$ true \\
        \hline
        TC-BR-02 & Ветвь 1 (false), 2.1 (true) & Граф с одной вершиной & Обход одной вершины \\
        \hline
        TC-BR-03 & Ветвь 2.1.1 (cycle) & Граф A$\rightarrow$B$\rightarrow$A & Цикл обнаружен \\
        \hline
        TC-BR-04 & Ветвь 2.1.2 (already visited) & Граф с общим потомком & Пропуск посещённой \\
        \hline
        TC-BR-05 & Ветвь 2.1.3 (process) & Линейный граф & Полный обход \\
        \hline
    \end{tabular}
\end{table}
\FloatBarrier

\section{Оптимизация тестовых случаев} \label{ch3:optimization}

После анализа разработанных тестовых случаев выявлено, что некоторые из них покрывают одинаковые сценарии. Это позволяет провести оптимизацию путём объединения тестов.

\subsection*{Объединённые тест-кейсы}

В таблице~\ref{tab:optimized_cases} представлены оптимизированные тестовые случаи с указанием покрываемых исходных тестов.

\begin{table}[htbp]
    \centering
    \caption{Оптимизированные тестовые случаи}
    \label{tab:optimized_cases}
    \begin{tabular}{|l|p{8cm}|}
        \hline
        \textbf{Оптимизированный TC} & \textbf{Покрывает} \\
        \hline
        TC-OPT-01: Пустой граф & TC-EP-01, TC-BVA-01, TC-DT-01, TC-BR-01 \\
        \hline
        TC-OPT-02: Одна вершина & TC-BVA-02, TC-BR-02 \\
        \hline
        TC-OPT-03: Линейный DAG & TC-EP-02, TC-BVA-04, TC-DT-02, TC-BR-05 \\
        \hline
        TC-OPT-04: Цикл & TC-EP-03, TC-DT-03, TC-BR-03 \\
        \hline
        TC-OPT-05: Невалидный JSON & TC-EP-04, TC-DT-05 \\
        \hline
        TC-OPT-06: Несуществующий файл & TC-DT-04 \\
        \hline
        TC-OPT-07: Два несвязных компонента & TC-BVA-03 \\
        \hline
        TC-OPT-08: Полный DAG & TC-BVA-05 \\
        \hline
        TC-OPT-09: Большой граф & TC-BVA-06 \\
        \hline
        TC-OPT-10: Общий потомок & TC-BR-04 \\
        \hline
        TC-OPT-11: Без выбора алгоритма & TC-DT-06 \\
        \hline
        TC-OPT-12: Сравнение алгоритмов & Интеграционный тест \\
        \hline
    \end{tabular}
\end{table}
\FloatBarrier

В результате оптимизации из 21 исходного тест-кейса получено 12 оптимизированных, что позволяет сократить трудозатраты на выполнение тестов без потери качества покрытия.

\section{Сравнительный анализ методов тестирования} \label{ch3:comparison}

\subsection*{Сводные таблицы характеристик}

\begin{table}[htbp]
    \centering
    \caption{Сравнительная таблица методов тестирования}
    \label{tab:comparison}
    \begin{tabular}{|l|c|c|p{2.5cm}|p{2.5cm}|}
        \hline
        \textbf{Метод} & \textbf{Test Cases} & \textbf{Покрытие вх. данных} & \textbf{Покрытие логики} & \textbf{Трудозатраты} \\
        \hline
        Equivalence Partitioning & 4 & Высокое & Низкое & Низкие \\
        \hline
        Boundary Value Analysis & 6 & Среднее & Низкое & Низкие \\
        \hline
        Decision Table & 6 & Высокое & Высокое & Средние \\
        \hline
        Branch Testing & 5 & Низкое & Высокое & Высокие \\
        \hline
        \textbf{ИТОГО} & \textbf{21} & --- & --- & --- \\
        \hline
    \end{tabular}
\end{table}
\FloatBarrier

\begin{table}[htbp]
    \centering
    \caption{Детальное сравнение техник тестирования}
    \label{tab:comparison_detailed}
    \small
    \begin{tabular}{|p{3cm}|c|c|c|c|}
        \hline
        \textbf{Критерий} & \textbf{EP} & \textbf{BVA} & \textbf{DT} & \textbf{Branch} \\
        \hline
        Обнаружение граничных дефектов & Низкое & Высокое & Среднее & Среднее \\
        \hline
        Применимость & Любые входы & Числ. данные & Комб. условий & Код \\
        \hline
    \end{tabular}
\end{table}
\FloatBarrier


\begin{table}[htbp]
    \centering
    \caption{Оценка трудозатрат по методам тестирования}
    \small
    \label{tab:effort}
    \begin{tabular}{|l|c|c|c|c|}
        \hline
        \textbf{Техника} & \textbf{Анализ} & \textbf{Создание} & \textbf{Поддер.} & \textbf{Итого} \\
        \hline
        Equivalence Partitioning & 1 ч & 1 ч & 0.5 ч & 2.5 ч \\
        \hline
        Boundary Value Analysis & 1 ч & 1.5 ч & 0.5 ч & 3 ч \\
        \hline
        Decision Table & 2 ч & 2 ч & 1 ч & 5 ч \\
        \hline
        Branch Testing & 3 ч & 2 ч & 1.5 ч & 6.5 ч \\
        \hline
        \textbf{Всего} & \textbf{7 ч} & \textbf{6.5 ч} & \textbf{3.5 ч} & \textbf{17 ч} \\
        \hline
    \end{tabular}
\end{table}
\FloatBarrier

\subsection*{Преимущества и недостатки методов}

\textbf{Equivalence Partitioning.}
\begin{itemize}
    \item \textbf{Преимущества:} минимум тестов, хорошее покрытие классов данных.
    \item \textbf{Недостатки:} может пропустить граничные ошибки.
\end{itemize}

\textbf{Boundary Value Analysis.}
\begin{itemize}
    \item \textbf{Преимущества:} эффективно находит граничные ошибки.
    \item \textbf{Недостатки:} применим не ко всем типам данных.
\end{itemize}

\textbf{Decision Table Testing.}
\begin{itemize}
    \item \textbf{Преимущества:} систематический, покрывает комбинации условий.
    \item \textbf{Недостатки:} экспоненциальный рост при многих условиях.
\end{itemize}

\textbf{Branch Testing.}
\begin{itemize}
    \item \textbf{Преимущества:} гарантирует покрытие кода.
    \item \textbf{Недостатки:} требует знания внутренней структуры (white-box).
\end{itemize}

\section{Выводы} \label{ch3:conclusion}

В данной главе разработана тестовая документация для приложения топологической сортировки с использованием четырёх методов: Equivalence Partitioning, Boundary Value Analysis, Decision Table Testing и Branch Testing.

Определено 16 проверяемых условий, распределённых по трём категориям: входные данные (8 условий), выходные данные (4 условия) и внутренняя обработка (4 условия).

Всего разработан 21 контрольный пример: 4 для Equivalence Partitioning, 6 для Boundary Value Analysis, 6 для Decision Table Testing и 5 для Branch Testing. После оптимизации количество тестовых случаев сокращено до 12 без потери качества покрытия.
