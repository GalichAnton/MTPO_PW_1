\chapter{Разработка тестовой документации} \label{ch3}

В данной главе представлена разработка тестовой документации для приложения топологической сортировки, реализующего алгоритмы DFS и Кана. Для каждого метода тестирования определены проверяемые условия (test cover items) и контрольные примеры (test cases). В завершение главы приведён сравнительный анализ использованных методов.

\section{Equivalence Partitioning (Разбиение на классы эквивалентности)} \label{ch3:ep}

Метод разбиения на классы эквивалентности основан на разделении входных данных на группы, для которых поведение системы должно быть одинаковым. Из каждого класса достаточно выбрать одного представителя для тестирования.

\subsection{Проверяемые условия}

В таблице~\ref{tab:ep_items} представлены выделенные классы эквивалентности для входных данных приложения.

\begin{table}[htbp]
    \centering
    \caption{Классы эквивалентности входных данных}
    \label{tab:ep_items}
    \begin{tabular}{|c|l|l|c|}
        \hline
        \textbf{ID} & \textbf{Параметр} & \textbf{Класс} & \textbf{Тип} \\
        \hline
        EP1 & Граф & Пустой & Valid \\
        \hline
        EP2 & Граф & DAG (ациклический) & Valid \\
        \hline
        EP3 & Граф & С циклом & Invalid \\
        \hline
        EP4 & Граф & Несвязный DAG & Valid \\
        \hline
        EP5 & JSON & Невалидный формат & Invalid \\
        \hline
    \end{tabular}
\end{table}

\subsection{Контрольные примеры}

На основе выделенных классов эквивалентности разработаны контрольные примеры, представленные в таблице~\ref{tab:ep_cases}.

\begin{table}[htbp]
    \centering
    \caption{Контрольные примеры для Equivalence Partitioning}
    \label{tab:ep_cases}
    \begin{tabular}{|c|c|p{5cm}|p{4cm}|}
        \hline
        \textbf{TC ID} & \textbf{Покрывает} & \textbf{Входные данные} & \textbf{Ожидаемый результат} \\
        \hline
        TC-EP1 & EP1 & vertices=[], edges=[] & [] \\
        \hline
        TC-EP2 & EP2 & vertices=[A,B,C], edges=[A→B, B→C] & [A,B,C] \\
        \hline
        TC-EP3 & EP3 & vertices=[A,B], edges=[A→B, B→A] & CycleException \\
        \hline
        TC-EP4 & EP4 & vertices=[A,B,X,Y], edges=[A→B, X→Y] & Любой корректный порядок \\
        \hline
        TC-EP5 & EP5 & "\{ invalid json" & JsonParseException \\
        \hline
    \end{tabular}
\end{table}

\section{Boundary Value Analysis (Анализ граничных значений)} \label{ch3:bva}

Метод анализа граничных значений направлен на тестирование значений на границах классов эквивалентности, поскольку именно на границах наиболее часто возникают ошибки.

\subsection{Проверяемые условия}

Граничные значения для параметров приложения представлены в таблице~\ref{tab:bva_items}.

\begin{table}[htbp]
    \centering
    \caption{Граничные значения параметров}
    \label{tab:bva_items}
    \begin{tabular}{|c|l|l|}
        \hline
        \textbf{ID} & \textbf{Параметр} & \textbf{Граница} \\
        \hline
        BV1 & Кол-во вершин & 0 \\
        \hline
        BV2 & Кол-во вершин & 1 \\
        \hline
        BV3 & Кол-во вершин & Большое (1000) \\
        \hline
        BV4 & Кол-во рёбер & Максимум для DAG: $\frac{n(n-1)}{2}$ \\
        \hline
        BV5 & Длина пути & Максимальная (цепочка $n-1$) \\
        \hline
    \end{tabular}
\end{table}

\subsection{Контрольные примеры}

Контрольные примеры для граничных значений представлены в таблице~\ref{tab:bva_cases}.

\begin{table}[htbp]
    \centering
    \caption{Контрольные примеры для Boundary Value Analysis}
    \label{tab:bva_cases}
    \begin{tabular}{|c|c|p{5cm}|p{4cm}|}
        \hline
        \textbf{TC ID} & \textbf{Покрывает} & \textbf{Входные данные} & \textbf{Ожидаемый результат} \\
        \hline
        TC-BV1 & BV1 & 0 вершин & [] \\
        \hline
        TC-BV2 & BV2 & 1 вершина & [v1] \\
        \hline
        TC-BV3 & BV3 & 1000 вершин (линейно) & Корректный порядок, $<1$с \\
        \hline
        TC-BV4 & BV4 & 4 вершины, 6 рёбер (полный DAG) & Корректный порядок \\
        \hline
        TC-BV5 & BV5 & Цепочка 10 вершин & [v1,v2,...,v10] \\
        \hline
    \end{tabular}
\end{table}

\section{Decision Table Testing (Таблица решений)} \label{ch3:dt}

Метод таблиц решений позволяет систематически рассмотреть все комбинации условий и соответствующих им действий системы.

\subsection{Проверяемые условия}

Для построения таблицы решений выделены следующие условия:
\begin{itemize}
    \item C1: Граф пустой?
    \item C2: Граф содержит цикл?
    \item C3: JSON валиден?
\end{itemize}

Таблица решений представлена в таблице~\ref{tab:dt_rules}.

\begin{table}[htbp]
    \centering
    \caption{Таблица решений}
    \label{tab:dt_rules}
    \begin{tabular}{|c|c|c|c|l|}
        \hline
        \textbf{Rule} & \textbf{C1} & \textbf{C2} & \textbf{C3} & \textbf{Действие} \\
        \hline
        R1 & Y & -- & Y & Вернуть [] \\
        \hline
        R2 & N & Y & Y & CycleException \\
        \hline
        R3 & N & N & Y & Вернуть порядок \\
        \hline
        R4 & -- & -- & N & JsonParseException \\
        \hline
    \end{tabular}
\end{table}

\subsection{Контрольные примеры}

Контрольные примеры на основе таблицы решений представлены в таблице~\ref{tab:dt_cases}.

\begin{table}[htbp]
    \centering
    \caption{Контрольные примеры для Decision Table Testing}
    \label{tab:dt_cases}
    \begin{tabular}{|c|c|l|l|}
        \hline
        \textbf{TC ID} & \textbf{Rule} & \textbf{Описание} & \textbf{Результат} \\
        \hline
        TC-DT1 & R1 & Пустой граф & [] \\
        \hline
        TC-DT2 & R2 & Граф с циклом & CycleException \\
        \hline
        TC-DT3 & R3 & Корректный DAG & Топологический порядок \\
        \hline
        TC-DT4 & R4 & Невалидный JSON & JsonParseException \\
        \hline
    \end{tabular}
\end{table}

\section{Branch Testing (Тестирование ветвей)} \label{ch3:branch}

Тестирование ветвей относится к методам структурного (white-box) тестирования и требует прохождения каждой ветви условных операторов в коде хотя бы один раз.

\subsection{Проверяемые условия}

Ключевые ветви алгоритмов DFS и Кана представлены в таблице~\ref{tab:br_items}.

\begin{table}[htbp]
    \centering
    \caption{Ветви для тестирования}
    \label{tab:br_items}
    \begin{tabular}{|c|l|l|l|}
        \hline
        \textbf{ID} & \textbf{Ветвь} & \textbf{True} & \textbf{False} \\
        \hline
        BR1 & graph.isEmpty() & Пустой граф & Непустой \\
        \hline
        BR2 & recStack.contains(v) & Цикл найден & Нет цикла \\
        \hline
        BR3 & !visited.contains(v) & Новая вершина & Уже посещена \\
        \hline
    \end{tabular}
\end{table}

\subsection{Контрольные примеры}

Контрольные примеры для покрытия ветвей представлены в таблице~\ref{tab:br_cases}.

\begin{table}[htbp]
    \centering
    \caption{Контрольные примеры для Branch Testing}
    \label{tab:br_cases}
    \begin{tabular}{|c|l|p{4cm}|l|}
        \hline
        \textbf{TC ID} & \textbf{Покрывает} & \textbf{Входные данные} & \textbf{Путь} \\
        \hline
        TC-BR1 & BR1-T & vertices=[] & isEmpty→true \\
        \hline
        TC-BR2 & BR1-F, BR2-T & A→B→A & Цикл обнаружен \\
        \hline
        TC-BR3 & BR1-F, BR2-F, BR3-T/F & A→B→C & Полный обход \\
        \hline
    \end{tabular}
\end{table}

\section{Сравнительный анализ методов тестирования} \label{ch3:comparison}

\subsection{Сводная таблица}

Результаты применения различных методов тестирования представлены в таблице~\ref{tab:comparison}.

\begin{table}[htbp]
    \centering
    \caption{Сравнительная таблица методов тестирования}
    \label{tab:comparison}
    \begin{tabular}{|l|c|c|l|l|}
        \hline
        \textbf{Метод} & \textbf{Test Items} & \textbf{Test Cases} & \textbf{Покрытие} & \textbf{Трудозатраты} \\
        \hline
        Equivalence Partitioning & 5 & 5 & Функциональное & Низкие \\
        \hline
        Boundary Value Analysis & 5 & 5 & Граничные случаи & Низкие \\
        \hline
        Decision Table & 4 & 4 & Комбинации условий & Средние \\
        \hline
        Branch Testing & 3 & 3 & Структурное & Высокие \\
        \hline
        \textbf{ИТОГО} & \textbf{17} & \textbf{17} & --- & --- \\
        \hline
    \end{tabular}
\end{table}

\subsection{Преимущества и недостатки методов}

\textbf{Equivalence Partitioning.}
\begin{itemize}
    \item \textbf{Преимущества:}
    \begin{itemize}
        \item Значительное сокращение количества тестов при сохранении качества покрытия.
        \item Логичная группировка входных данных, упрощающая анализ.
        \item Простота создания и поддержки тестовых сценариев.
        \item Эффективен на ранних этапах тестирования для быстрого покрытия основных сценариев.
    \end{itemize}
    \item \textbf{Недостатки:}
    \begin{itemize}
        \item Возможность пропуска ошибок на границах классов эквивалентности.
        \item Субъективность при выделении классов эквивалентности.
        \item Не учитывает комбинации нескольких параметров.
        \item Может быть недостаточным для комплексного тестирования бизнес-логики.
    \end{itemize}
\end{itemize}

\textbf{Boundary Value Analysis.}
\begin{itemize}
    \item \textbf{Преимущества:}
    \begin{itemize}
        \item Высокая эффективность в обнаружении распространённых ошибок типа «off-by-one».
        \item Отличное дополнение к методу Equivalence Partitioning.
        \item Фокусировка на наиболее вероятных местах возникновения дефектов.
        \item Простота применения для числовых диапазонов и упорядоченных данных.
    \end{itemize}
    \item \textbf{Недостатки:}
    \begin{itemize}
        \item Ограниченная применимость к нечисловым и неупорядоченным данным.
        \item Требует чёткого понимания границ параметров системы.
        \item Не покрывает комбинации граничных значений разных параметров.
        \item Может быть избыточным при неявно заданных границах.
    \end{itemize}
\end{itemize}

\textbf{Decision Table Testing.}
\begin{itemize}
    \item \textbf{Преимущества:}
    \begin{itemize}
        \item Систематическое покрытие всех значимых комбинаций условий.
        \item Наглядность представления сложной бизнес-логики.
        \item Исключение случайных пропусков в тестировании.
        \item Позволяет выявить противоречия и неполноту требований.
    \end{itemize}
    \item \textbf{Недостатки:}
    \begin{itemize}
        \item Экспоненциальный рост числа правил при увеличении количества условий ($2^n$ комбинаций).
        \item Трудоёмкость создания и поддержки таблиц решений для сложных систем.
        \item Требует формализации и чёткого понимания бизнес-логики.
        \item Может приводить к избыточному количеству тестовых случаев.
    \end{itemize}
\end{itemize}

\textbf{Branch Testing.}
\begin{itemize}
    \item \textbf{Преимущества:}
    \begin{itemize}
        \item Гарантирует прохождение всех ветвей кода алгоритма.
        \item Позволяет выявить «мёртвый» (недостижимый) код.
        \item Обеспечивает объективную метрику покрытия кода.
        \item Эффективен для проверки сложных условных конструкций.
    \end{itemize}
    \item \textbf{Недостатки:}
    \begin{itemize}
        \item Требует доступа к исходному коду (white-box подход).
        \item Не гарантирует проверки всех функциональных требований.
        \item Высокая трудоёмкость разработки тестов для достижения полного покрытия.
        \item Не учитывает комбинации путей выполнения программы.
        \item Может быть недостаточным для обнаружения ошибок в бизнес-логике.
    \end{itemize}
\end{itemize}

\section{Выводы} \label{ch3:conclusion}

В данной главе разработана тестовая документация для приложения топологической сортировки с использованием четырёх методов: Equivalence Partitioning, Boundary Value Analysis, Decision Table Testing и Branch Testing. Всего разработано 17 проверяемых условий и 17 контрольных примеров.
